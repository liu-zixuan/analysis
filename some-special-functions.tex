\section{Some Special Functions}
    \subsection{Power series}
    \begin{theo}
        Suppose the series $\sum_{n=0}^\infty c_n x^n (x \in \R)$ converges for $|x| < R$ and define
        \begin{equation}
            f(x) = \sum_{n=0}^\infty c_n x^n \ \ \ \ |x| < R
        \end{equation}
        
        Then
        \begin{enumerate}
            \item $\sum_{n=0}^\infty c_n x^n (x \in \R)$ converges uniformly on $[-R+\epsilon, R-\epsilon] \ \ \forall \epsilon > 0$.

            \item $f$ is differentiable.
            \begin{equation}
                f'(x) = \sum_{n=0}^{\infty} n c_n x^{n-1}
            \end{equation}
        \end{enumerate}
    \end{theo}

    \begin{cor}
        \begin{equation}
            f^{(k)}(x) = \sum_{n=k}^{\infty} n(n-1)\cdots (n-k+1) c_n x^{n-k}
        \end{equation}
        In particular, $f^{(k)}(0) = k!c_k$.
    \end{cor}

    \begin{theo}
        If $\sum_{n=0}^{\infty} c_n$ converges, put
        \begin{equation}
            f(x) = \sum_{n=0}^\infty c_n x^n \ \ \ \ -1 < x < 1
        \end{equation}
        Then
        \begin{equation}
            \lim_{x \to 1} f(x) = \sum_{n=0}^{\infty} c_n
        \end{equation}
    \end{theo}

    \begin{theo}
        Extenstion of Taylor's theorem
        \begin{equation}
            f(x) = \sum_{n=0}^\infty \frac{f^{(n)}(a)}{n!} (x-a)^n \ \ \ \ |x-a| < R - a
        \end{equation}
    \end{theo}

    \begin{theo}
        Suppose
        \begin{equation}
            \sum_{n=0}^\infty a_n x^n = \sum_{n=0}^\infty b_n x^n \ \ \ \ x \in E
        \end{equation}

        If $E$ has a limit point, then $a_n = b_n \ \forall n \in \N$.
    \end{theo}

    \subsection{Elementary Functions}
    \begin{defi}
        Define 
        \begin{equation}
            E(z) = \sum_{n=0}^{\infty} \frac{z^n}{n!} \ \ \ \ z \in \C
        \end{equation}
    \end{defi}
    \begin{theo}
        Some properties of $E(z)$.
        \begin{enumerate}
            \item $E(z+w) = E(z)E(w)$
            \item $E'(z) = E(z)$
        \end{enumerate}
    \end{theo}

    Suppose we have defined function $x^n (n \in \N^*, x \in \R)$ and $\sqrt[n]{x} (n \in \N^*, x \in \R^+)$. Next we define the exponential function $e^x (x \in \R)$ naturally with the function $E(x)$.
    \begin{itemize}
        \item Since $E(0) = 1$, define $e^0 = 1$.
        \item Let $n,m \in \N^*$
        \begin{equation}
            \left[E \left(\frac{n}{m}\right) \right]^m = E(n) = e^n
        \end{equation}
        Thus define
        \begin{equation}
            e^{\frac{n}{m}} = \sqrt[m]{e^n}
        \end{equation}
        \item Let $p \in \Q$
        \begin{equation}
            E(-p) = \frac{1}{E(p)} = \frac{1}{e^p}
        \end{equation}
        Thus define
        \begin{equation}
            e^{-p} = \frac{1}{e^p}
        \end{equation}
        \item Let $x \in \R \backslash \Q$
        \begin{equation}
            E(x) = \sup \{E(p) : p < x, p \in \Q \} = \sup \{e^p : p < x, p \in \Q \}
        \end{equation}
        Thus define
        \begin{equation}
            e^x = \sup \{e^p : p < x, p \in \Q \}
        \end{equation}
    \end{itemize}

    Since $E \in \mathscr{C}'(\R)$ and $\forall x \in \R \ E(x) \neq 0$, $E$ has an inverse function $L \in \mathscr{C}'(\R^+): \R^+ \to \R$. We use $\ln(x)$ to denote this function.

    \begin{defi}
        Exponential function with arbitrary base
        \begin{equation}
            a^x = e^{x \ln a} \ \ \ \ a \in \R^+, x \in \R
        \end{equation}
    \end{defi}

    \begin{defi}
        (Trigonometric Functions) 
        \begin{equation}
            \cos(x) = \frac{1}{2}[E(ix)+E(-ix)]
        \end{equation}
        \begin{equation}
            \sin(x) = \frac{1}{2i}[E(ix)-E(-ix)]
        \end{equation}
    \end{defi}

    \begin{theo}
        Derivatives elementary functions
        \begin{enumerate}
            \item Let $a \in \R^+, x \in \R$
            \begin{equation}
                \left( a^x \right)' = \left[ e^{x \ln a} \right]' = a^x \ln a
            \end{equation}

            \item Let $x \in \R^+, \alpha \in \R$,
            \begin{equation}
                \left(x^{\alpha}\right)' = \left[ e^{\alpha \ln x} \right]' = \alpha x^{\alpha-1} 
            \end{equation}

            \item Let $x \in \R$
            \begin{equation}
                \sin'(x) = \cos(x) \ \ \ \ \cos'(x) = -\sin(x)
            \end{equation}
        \end{enumerate} 
    \end{theo}

    \subsection{Fourier Series}
    \begin{defi}
        Let $\{ \phi_n \}$ be a sequence of complex functions on $[a,b]$. $\{ \phi_n \}$ is said to be \textbf{orthonormal} if 
        \begin{equation}
            \int_a^b \phi_n(x) \overline{\phi_m(x)} \dif x = \begin{cases}
                0 & m \neq n \\
                1 & m = n
            \end{cases}
        \end{equation}

        If $\{ \phi_n \}$ is orthonormal on $[a,b]$, and
        \begin{equation}
            c_n = \int_a^b f(t) \overline{\phi_n(t)} \dif t
        \end{equation}
        We call the following series the Fourier series of $f$ relative to $\{ \phi_n \}$
        \begin{equation}
            f(x) \sim \sum c_n \phi_n (x)
        \end{equation}
    \end{defi}

    \begin{theo}
        Let $\{ \phi_n \}$ is orthonormal on $[a,b]$. $f(x) \sim \sum c_n \phi_n (x)$.
        \begin{equation}
            s_n(x) = \sum_{m=1}^n c_m \phi_m (x)
        \end{equation}
        \begin{equation}
            t_n(x) = \sum_{m=1}^n \gamma_m \phi_m (x)
        \end{equation}
        Then
        \begin{equation}
            \int_a^b | f-s_n |^2 \dif x \leq \int_a^b | f-t_n |^2 \dif x
        \end{equation}
        with equality iff $\gamma_m = c_m \ (m=1,\cdots,n)$.
    \end{theo}

    \begin{theo}
        (Bessel inequality) Let $\{ \phi_n \}$ is orthonormal on $[a,b]$. $f(x) \sim \sum c_n \phi_n (x)$.
        \begin{equation}
            \sum |c_n|^2 \leq \int_a^b |f(x)|^2 \dif x
        \end{equation}
    \end{theo}
